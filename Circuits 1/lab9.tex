\documentclass{article}



\begin{document}


\begin{table}[h]
\begin{center}
\label{code:resis}
\caption{Measured Gain Values}
\begin{tabular}{c|c|c|c}
Variable & Measured Gain Value $A_{v,x}$ & $A_{v,e}$ & Percent Differences 
\\\hline
$A_{v,a}$ & $-6.92$ & $-6.583$ & 5.119\%  \\
$A_{v,b}$ & $-5.64$ & $-6.583$ & 14.3248\%  \\
$A_{v,c}$ & $-6.714$ & $-6.583$ & 1.98997\%  \\

\end{tabular}
\end{center}

\end{table}

Part B had high enough voltages that clipping occurred resulting in the amplification 


$$V_{m,max,exp} = 2.105$$
$$V_{m,max}= 2.25V$$
A large contribution to why this could be lower is that the resistor $R_f$ was lower than expected. It was the best one to be found.


The $V_o(t)$ for the first and third setup were close to predicted, but the second was very far off. The predicted value for $V_o(t)$ is $-6.666$, were the second setup had a $V_o(t)$ magnitude of $-16.4575$. 

The assumption that it is an ideal op-amp is a ok one. There a a few inconsistencies here or there but for the most part it would work. Most of the errors are with in 10\% meaning predictability is consistent.
\end{document}